% Chapter Template

\chapter{Literature Review} % Main chapter title

\label{Chapter2} % Change X to a consecutive number; for referencing this chapter elsewhere, use \ref{ChapterX}

%----------------------------------------------------------------------------------------
%	SECTION 1
%----------------------------------------------------------------------------------------

\section{Social Navigation in Robotics}

%-----------------------------------
%	SUBSECTION 1
%-----------------------------------
\subsection{Definition and Importance}
\cite{riosmartinez2015proxemics}\ gave a compact description of socially-aware navigation: 
\emph{``Socially-aware navigation is the strategy exhibited by a social robot that identifies 
and follows social conventions (in terms of management of space) to preserve a comfortable 
interaction with humans. The resulting behavior is predictable, adaptable, and easily 
understood by humans.''} This definition implies that, from the robot's point of view, 
humans are no longer perceived only as dynamic obstacles but also as social entities.

The importance of social navigation is paramount for robots (\cite{KRUSE20131726}) intended to operate in human-centric 
environments such as homes, hospitals, shopping malls, offices, and public spaces. 
As robots become increasingly integrated into our daily lives, their ability to interact seamlessly 
and naturally with humans is crucial for their acceptance and widespread adoption. Poor social 
navigation can lead to discomfort, anxiety, inefficiency, and even safety hazards for humans. 
Conversely, robots capable of navigating socially appropriately can enhance human productivity, 
provide assistance in various tasks, and improve overall quality of life. Furthermore, in 
applications like assistive robotics and healthcare, the ability of a robot to navigate in 
close proximity to individuals, while maintaining their comfort and safety, is fundamental (\cite{MOLLERsurvey}).


%-----------------------------------
%	SUBSECTION 2
%-----------------------------------

\subsection{Key Challenges}
\label{subsec:key_challenges}
Developing robust social navigation capabilities in robots presents several key challenges:

\begin{itemize}
    \item \textbf{Human Behavior Prediction:} Humans are inherently unpredictable. Their motion 
    patterns are influenced by a multitude of factors including their goals, intentions, 
    emotions, social context, and cultural background. Accurately predicting human trajectories 
    and intentions is a significant challenge, requiring sophisticated models that can capture 
    the nuances of human behavior.
    \item \textbf{Social Norms and Etiquette:} Navigating social environments requires adherence 
    to a complex set of implicit social norms and etiquette. These norms can vary across cultures 
    and situations. Robots need to understand and respect these norms, such as maintaining 
    appropriate personal space, avoiding sudden or erratic movements, and yielding to pedestrians 
    in certain situations.
    \item \textbf{Uncertainty and Dynamic Environments:} Human-populated environments are inherently 
    dynamic and uncertain. People may change their direction or speed unexpectedly, form groups, 
    or engage in interactions that affect robot navigation. Robots must be able to perceive and 
    react to these dynamic changes in real-time while maintaining their navigation goals.
    \item \textbf{Computational Complexity:} Implementing sophisticated models for human behavior 
    prediction, social norm understanding, and real-time adaptation can be computationally demanding. 
    Developing efficient algorithms that can run on robot platforms with limited computational 
    resources is a crucial challenge.
    \item \textbf{Evaluation Metrics and Benchmarking:} Defining appropriate metrics to evaluate 
    the social acceptability and effectiveness of robot navigation is challenging. Establishing 
    standardized benchmarks and simulation environments is necessary to facilitate the comparison 
    and progress of different social navigation approaches.
\end{itemize}

%----------------------------------------------------------------------------------------
%	SECTION 2
%----------------------------------------------------------------------------------------

\section{Proxemics and Human-Robot Interaction}


%-----------------------------------
%	SUBSECTION 1
%-----------------------------------

\subsection{Proxemics Theory}
Edward Hall's theory of proxemics \cite{proxemicstheoryHall} suggests
that people will maintain differing degrees of personal distance depending on the 
social setting and their cultural backgrounds. 
\begin{itemize}
    \item \textit{Intimate space} - the clossest ``bubble'' of space surrounding a person.
    Entry into this space is acceptable only for the closest friends and intimates.
    \item \textit{Social and consultative spaces} - the spaces in which people feel 
    comfortable conducting routine social ineractions with acquaintances as well as 
    strangers.
    \item \textit{Public space} - the area of space beyond which people will perceive 
    iteractions as impersonal and relatively anonymous.
\end{itemize}
The main contribution of Hall's Proxemics into path planning consists of providing a 
framework to build social maps, i.e. dynamic maps in which humans are perceived 
as obstacles following the definition of Hall's personal space.   
The work by \cite{henkel} evaluates different distance strategies by how 
they affect the human's perception of the robot's likeability, intelligence and submissiveness.
%-----------------------------------
%	SUBSECTION 2
%-----------------------------------

\subsection{Application in Robotics}
Proxemics has been employed in robotics to guide the development of spatially aware path 
planning. Robots use proxemic principles to maintain comfortable distances from humans, 
avoid intrusions into personal zones, and adjust their behavior based on environmental 
context. Several implementations integrate proxemic rules into costmaps and behavior 
trees to ensure adherence to social comfort zones, increasing user satisfaction and 
perceived safety. Proxemic-aware navigation also supports differentiated behaviors 
depending on robot intent — for example, service robots vs. delivery robots — providing 
richer HRI experiences
%----------------------------------------------------------------------------------------
%	SECTION 3
%----------------------------------------------------------------------------------------

\section{Simulation Platforms for Social Navigation}
\cite{Helbing_1995}, show that pedestrian motion can be described by a simple 
social force model for individual pedestrian behavior. The social force model is an essential 
component in many platforms including Arena-rosnav \cite{arenabench}, HuNavSim \cite{hunavsimros2human}
etc. to simulate the pedestrian movements. 
The navigation comprises of path planners and costmaps. There are two types of planners - global 
planner determines a path from the current location to the goal location, and a local planner 
follows the global path. Costmaps are created using static maps and real-time data from onboard
sensors. \\

\cite{MPCwithsfm}, use A* global planner and an MPC, with a detailed cost 
function to achieve advanced social navigation capabilities with the help of SMPC (Social 
Model Predictive Control) software stack. They leverage the predicitvity of MPC and the 
reactivity of SFM, modelling the pedestrian motion.\\

\cite{chen2018sociallyawaremotionplanning}, presented SA-CADRL (Socially Aware Collision 
Avoidance with Deep Reinforcement Learning) to explain/induce socially aware behaviors in a RL 
framework. They generalized to multiagent ($n > 2$) scenarios through developing symmetrical neural
network structure, and demonstrated on robotic hardware autonomous navigation at human walking 
speed in a pedestrian-rich environment. \\

There exist various other approaches based on traditional algoritms as well as novel 
Neural Network, Deep Reinforcement Learning etc. In the next section, several simulation platforms,
their advantages and disadvantages are described. 

%-----------------------------------
%	SUBSECTION 1
%-----------------------------------

\subsection{Overview of Existing Platforms}
This section focuses on capabilities, features, and limitations of each environment for 
making specific recommendations for implementation purposes. \\

\textit{Habitat-Sim} is a flexible, high performance 3D simulator with a focus on embodied AI 
research including navigation tasks \cite{habitat}, \cite{szot2022habitat20traininghome}. 
It is capable of running thousands of simulations in parallel, with photo-realistic 3D 
environments from real-world scans, semantic scene understanding and support for multiple 
sensors (RGB, depth, semantic segmentation). It is however limited by a lack of in-built 
social navigation features, human motion models can be integrated however it requires some
technical understanding of AI and has a steep learning curve. \\

\textit{SEAN 2.0} is specifically designed for social navigation research with emphasis on 
human behavior modeling \cite{tsoi2022sean2}. It is a high fidelity, extensible, and open-source simulation platform for fair evaluation of social navigation algorithms. 
Environments correspond to the physical, static elements in a scenario in Unity. \textit{SEAN 2.0}
\footnote{\href{https://sean.interactive-machines.com/}{https://sean.interactive-machines.com/}}
includes warehouss, lab, and outdoor environments from \textit{SEAN 1.0} with annotations 
for new pedestrian behaviours. It also provides numerous evalutation metrics including path 
efficiency, path irregularity, completed, totol time etc. It is limited by the simulation 
backend options as it limited to Unity.  \\

\textit{HuNavSim} focuses specifically on realistic human navigation behavior modelling \cite{hunavsimros2human}.
It is a new open-source software library used to simulate human navigation behaviors. The tool,
programmed under the new ROS 2 framework, can be employed to control the human agents of 
different general robotics simulators. It utilizes a Social Costmap Layer (a custom ROS 2 version 
of the social navigation layers implemented in ROS 1
\footnote{\href{https://github.com/robotics-upo/nav2_social_costmap_plugin}
{https://github.com/robotics-upo/nav2\_social\_costmap\_plugin}})
It provides a wide range of metrics for various scenarious, however it is limited by the number of 
available planners and lacks documentation which makes it a difficult choice for educational 
applications. \\

\textit{Arena-rosnav} is an open-source  modular benchmark environment built on ROS that specifically targets
socially aware navigation \cite{arenarosnav}. It provides support a total of 15 navigation planners (see table: \ref{tab:navigation_planners})
which include classic, hybrid and learning-based planners. It (Arena Rosnav 3.0)\footnote{\href{https://3.arena-rosnav.org/}{https://3.arena-rosnav.org/}} provides 
support for several both 2D and 3D simulators including Flatland, Rviz, Gazebo, Unity and provides 
an interface to integrate other simulation softwares, worlds like  
\textit{Hospital, Canteen, Campus, Factory and Warehouse} are supported in Gazebo and 
\textit{Hospital, Restaurant School, Japanese Garden and Warehouse} are supported in Unity,
additionally users can add or create new worlds. Multiple robots including 
\textit{tutlebot3-burger, jackal, ridgeback, agv-ota, tiago, robotino, youbot, turtlebot3\_waffle\_pi etc.} 
are supported and users can add more easily. It's limitations however are that it is a little above 
50 GB in size and computationally intensive. 

\noindent TODO: fix appearance of citations.
\begin{table}[ht]
    \centering
    \caption{Available Navigation Planners}
    \label{tab:navigation_planners}
    \begin{tabularx}{\textwidth}{X X X}
        \hline
        \textbf{Classic}    & \textbf{Hybrid}                       & \textbf{Learning-based} \\
        \hline
        TEB~\cite{teb}      & Applr~\cite{applr}                    & ROSNavRL~\cite{arenarosnav} \\
        DWA~\cite{dwa}      & LfLH~\cite{lflh}                      & RLCA~\cite{rlca} \\
        MPC~\cite{mpc}      & Dragon~\cite{dragontrail}             & Crowdnav~\cite{crowdnav} \\
        Cohan~\cite{cohan}  & TRAIL~\cite{dragontrail}              & SARL~\cite{sarl} \\
                            &                                       & Arena~\cite{arena} \\
                            &                                       & CADRL~\cite{cadrl} \\
                            &                                       & Navrep~\cite{navrep} \\
        \hline
    \end{tabularx}
\end{table}
   
%-----------------------------------
%	SUBSECTION 2
%-----------------------------------

\subsection{Comparison and Suitability for Education}
Simulation tools differ in usability, fidelity, extensibility, and educational value:
\begin{itemize}
\item \textbf{Habitat-Sim} offers high visual fidelity but is difficult for beginners due 
to limited social behavior modules.
\item \textbf{SEAN 2.0} provides tailored environments for social navigation and good 
evaluation metrics, making it suitable for advanced education.
\item \textbf{HuNavSim} is lightweight and ROS 2 compatible but suffers from poor 
documentation.
\item \textbf{Arena-Rosnav} is ideal for research and advanced robotics courses, providing 
a wide range of planners and robot models, but requires high computational resources.
\end{itemize}
This thesis utilizes Arena-Rosnav, the reasons for selecting it are discussed in section 
\ref{sec:platform_selection_and_justification}
%----------------------------------------------------------------------------------------
%	SECTION 4
%----------------------------------------------------------------------------------------

\section{Summary of Gaps and Research Opportunities}

While considerable progress has been made, several gaps persist:
\begin{itemize}
\item Lack of universal benchmarks for social navigation evaluation.
\item Insufficient modeling of nuanced social behaviors (e.g., group dynamics, cultural norms).
\item Limited cross-simulator compatibility.
\item High barrier to entry due to complex setups or resource demands.
\item Sparse integration between proxemics theory and learning-based planners.
\end{itemize}

Opportunities exist in developing:
\begin{itemize}
\item Lightweight, user-friendly simulation tools for education.
\item Integrative frameworks combining proxemics.
\end{itemize}