% Chapter Template

\chapter{Methodology} % Main chapter title

\label{Chapter4} % Change X to a consecutive number; for referencing this chapter elsewhere, use \ref{ChapterX}

Building upon the social navigation simulation environment described in Chapter 
\ref{Chapter3}, this chapter outlines the methodological framework used 
to evaluate both the technical efficacy of social navigation algorithms and their 
educational value. Our methodology addresses the dual research objectives of 
advancing social navigation algorithm performance and developing effective 
educational approaches for teaching social navigation concepts.
%----------------------------------------------------------------------------------------
%	SECTION 1
%----------------------------------------------------------------------------------------

\section{Experiment Design}\label{sec:experiment_design}
The experimental design follows a systematic approach that isolates specific 
aspects of social navigation while maintaining sufficient environmental complexity 
to yield meaningful results. We employed both controlled comparative experiments 
for algorithm evaluation and structured learning sequences for educational 
assessment.
%-----------------------------------
%	SUBSECTION 1
%-----------------------------------

\subsection{Scenario Configuration}
\label{subsec:scenario_configuration}
To ensure experimental rigor, we developed a standardized approach to scenario 
configuration that maintains consistency across experimental trials while allowing 
for systematic parameter variation:
\begin{itemize}
    \item \textbf{Scenario Definition Protocol} -- Each experimental scenario is 
    defined through a hierarchical YAML configuration that specifies environment 
    geometry, human agent parameters, robot configuration, and navigation algorithm 
    settings. This approach enables precise replication of experimental conditions.
    \item \textbf{Initialization Procedures} -- Robot and human agent positions are 
    initialized using consistent starting configurations with controlled randomization 
    where appropriate. For each scenario variant, we defined standard starting regions 
    rather than exact positions, allowing natural variation while maintaining comparable 
    interaction conditions.
    \item \textbf{Trial Duration and Termination Criteria} -- Experimental trials run 
    until one of several termination conditions is met: successful navigation to goal, 
    collision event, timeout (120 seconds), or deadlock detection (no progress for 15 seconds). 
    This approach ensures comparable data collection across algorithm variants.
    \item \textbf{Scenario Progression} -- Experiments follow a progression from baseline 
    scenarios (single human, simple environment) to complex scenarios (multiple humans, 
    challenging environmental features). This structured progression supports both 
    incremental algorithm testing and pedagogical sequencing.
\end{itemize}
For each of the core scenarios described in Chapter \ref{Chapter3} (corridor and 
corner navigation), we developed specific configuration variants that 
systematically increase in complexity:
\begin{table}[ht]
    \centering
    \caption{Scenario Progression for Experimental Evaluation}
    \label{tab:scenario_progression}
    \begin{tabularx}{\textwidth}{X X X}
        \hline
        \textbf{Complexity Level}           & \textbf{Corridor Scenario}                & \textbf{Corner Scenario} \\
        \hline
        Basic                               & Single oncoming human, centered approach  & Single human around corner, moderate speed \\
        Intermediate                        & Multiple humans, bidirectional flow       & Multiple humans with varied approach speeds \\
        Advanced                            & Group formations with social structures   & Blind corner with distracted humans \\
        Complex                             & High-density crowd with bottlenecks       & Multiple corners with crossing patterns \\
        \hline
    \end{tabularx}
\end{table}
Through Arena-Rosnav's scenario generation capabilities, we programmatically created these
 scenarios with consistent parameters while allowing controlled variation in human movement 
 patterns. This approach yields statistically meaningful results while capturing the natural 
 variability of human-robot interactions.

%-----------------------------------
%	SUBSECTION 2
%-----------------------------------

\subsection{Variables and Parameters}
\label{subsec:variables_parameters}
The experimental design incorporates independent variables (controlled by the experimenter), 
dependent variables (measured outcomes), and controlled variables (held constant 
across experimental conditions).

%-----------------------------------
%	SUBSECTION 2
%-----------------------------------

\subsubsection{Independent Variables}
\label{subsubsec:independent_variables}
We systematically manipulated the following independent variables across experimental trials:
\begin{itemize}
    \item \textbf{Navigation Algorithm Configuration}
    \begin{itemize}
        \item Social cost function parameters (personal space radius: 0.45m-1.2m)
        \item Proxemic layer weighting (0.3-1.0 relative to obstacle costs)
        \item Planning horizon duration (1.0s-3.5s for local planning)
        \item Goal-directed vs. social compliance behavior weighting
    \end{itemize}
    \item \textbf{Human Density and Distribution}
    \begin{itemize}
        \item Number of humans (1-12)
        \item Spatial distribution patterns (uniform, clustered, directional)
        \item Group formation configurations (individuals, dyads, larger groups)
    \end{itemize}
    \item \textbf{Human Behavior Models}
    \begin{itemize}
        \item Attention levels (attentive, distracted, unaware)
        \item Cooperation levels (cooperative, neutral, uncooperative)
        \item Movement predictability (highly predictable to erratic)
    \end{itemize}
    \item \textbf{Environmental Constraints}
    \begin{itemize}
        \item Corridor width (1.8m-3.5m)
        \item Corner angle (70°-110°)
        \item Obstacle density and placement
    \end{itemize}
\end{itemize}
These variables were manipulated using Arena-Rosnav's parameter configuration system, which 
allows systematic parameter sweeps across predefined ranges. For educational experiments, 
a subset of these variables was exposed to students through simplified interfaces appropriate 
to their learning stage.

%-----------------------------------
%	SUBSECTION 2
%-----------------------------------
\subsubsection{Dependent Variables}
\label{subsubsec:dependent_variables}
We measured several categories of dependent variables to evaluate both technical performance 
and social compliance:
\begin{itemize}
    \item \textbf{Navigation Efficiency Metrics}
    \begin{itemize}
        \item Path length ratio (actual/optimal)
        \item Navigation time ratio (actual/optimal)
        \item Smoothness of trajectory (jerk analysis)
        \item Computational efficiency (planning cycle time)
    \end{itemize}
    \item \textbf{Safety Metrics}
    \begin{itemize}
        \item Minimum separation distance to humans
        \item Collision count and near-miss frequency
        \item Time spent in human personal/social spaces
        \item Velocity modulation near humans
    \end{itemize}
    \item \textbf{Social Compliance Metrics}
    \begin{itemize}
        \item Adherence to social conventions (right/left passing, queuing)
        \item Human trajectory disturbance (deviation from preferred paths)
        \item Predictability of robot movement (human-reported comfort)
        \item Social signal appropriateness (speed, direction changes)
    \end{itemize}
    \item \textbf{Educational Metrics} (for student experiments)
    \begin{itemize}
        \item Conceptual understanding assessment scores
        \item Parameter tuning effectiveness
        \item Algorithm selection appropriateness
        \item Experimental design quality
    \end{itemize}
\end{itemize}
These metrics were collected automatically through Arena-Rosnav's integrated data collection system, 
which records comprehensive trajectory, planning, and interaction data for each experimental trial.

%-----------------------------------
%	SUBSECTION 2
%-----------------------------------

\subsubsection{Controlled Variables}
\label{subsubsec:controlled_variables}
To ensure experimental validity, we held the following variables constant 
across experimental conditions:
\begin{itemize}
    \item \textbf{Robot Platform Configuration}
    \begin{itemize}
        \item Physical dimensions (0.5m diameter circular footprint)
        \item Sensor configuration (270° laser scanner, 4m range)
        \item Maximum velocity constraints (0.5m/s linear, 1.2rad/s angular)
        \item Actuator response characteristics (constant acceleration model)
    \end{itemize}
    \item \textbf{Simulation Parameters}
    \begin{itemize}
        \item Physics engine configuration (step size, solver iterations)
        \item Sensor noise models (constant parameters)
        \item Environmental conditions (lighting, floor texture)
        \item Simulation time scaling (real-time execution)
    \end{itemize}
\end{itemize}
These controlled variables ensure that observed differences in experimental outcomes can be 
attributed to the independent variables rather than uncontrolled factors in the simulation 
environment.


%----------------------------------------------------------------------------------------
%	SECTION 2
%----------------------------------------------------------------------------------------
\section{Data Collection Methods}
\label{sec:data_collection}
Our data collection approach combines automated quantitative measurements, qualitative 
assessments, and educational evaluations to provide a comprehensive view of social 
navigation performance.

-----------------------------------
%	SUBSECTION 1
%-----------------------------------

\subsection{Automated Measurement System}
\label{subsec:automated_measurement}
We extended Arena-Rosnav's data collection capabilities with specialized components 
for social navigation metrics:
\begin{itemize}
        \item \textbf{Trajectory Recording System} -- Captures time-indexed position, 
        velocity, and acceleration data for the robot and all human agents at 10Hz. 
        This data forms the foundation for trajectory analysis, proximity calculations, 
        and social compliance evaluation.
        \item \textbf{Planning Process Instrumentation} -- Records internal planning data 
        including considered paths, rejected trajectories, cost distribution maps, and 
        computation time. This instrumentation provides insight into the navigation 
        algorithm's decision-making process.
        \item \textbf{Proxemic Violation Detection} -- Implements a zonal model of 
        human personal space with concentric regions (intimate: 0-0.45m, personal: 0.45-1.2m, 
        social: 1.2-3.6m) and continuously monitors for zone penetration events, recording 
        duration and penetration depth.
        \item \textbf{Social Convention Monitoring} -- Detects adherence to environment-specific 
        social conventions including preferred passing side, yielding behavior, and queuing 
        dynamics. This monitoring system uses geometric analysis of trajectories to identify 
        convention adherence or violation.
        \item \textbf{Environmental Context Tracking} -- Captures contextual information 
        including location type (corridor, corner, open space), density conditions, and 
        environmental constraints. This contextual data supports nuanced analysis of 
        navigation behavior across different environmental conditions.
    \end{itemize}
Data collection runs as a distributed system integrated with ROS, with dedicated nodes 
for different measurement aspects. All data is time-synchronized through the ROS clock 
and stored in both ROS bag format (for detailed analysis) and processed CSV files 
(for statistical analysis).

%-----------------------------------
%	SUBSECTION 2
%-----------------------------------

\subsection{Qualitative Assessment}
\label{subsec:qualitative_assessment}
In addition to quantitative measurements, we conducted qualitative assessments of navigation 
behavior:
\begin{itemize}
    \item \textbf{Expert Evaluation Protocol} -- We developed a structured protocol for 
    expert assessment of robot social navigation behavior. This protocol includes rating 
    scales for naturalness, predictability, appropriateness, and overall social intelligence, 
    applied to recorded navigation episodes.
    \item \textbf{Behavior Coding System} -- A formal coding system for categorizing specific 
    robot behaviors during human interactions, including avoidance initiation timing, trajectory 
    communication clarity, and recovery behavior appropriateness.
    \item \textbf{Critical Incident Analysis} -- Detailed examination of edge cases and failure 
    modes, documenting the circumstances and contributing factors to navigation breakdowns or 
    socially inappropriate behaviors.
\end{itemize}
Three domain experts with backgrounds in robotics, human-robot interaction, and social 
psychology independently evaluated a subset of experimental trials using these qualitative 
methods. Inter-rater reliability was assessed using Cohen's kappa coefficient, with a m
inimum threshold of $K > 0.7$ required for inclusion in the analysis.

%-----------------------------------
%	SUBSECTION 3
%-----------------------------------
\subsection{Educational Assessment}
\label{subsec:educational_assessment}
For the educational objectives of our research, we implemented a multi-faceted assessment 
approach:
\begin{itemize}
    \item \textbf{Pre/Post Knowledge Assessments} -- Validated instruments measuring student 
    understanding of key social navigation concepts before and after using the simulation 
    environment. These assessments include both technical knowledge (algorithm operation, 
    parameter effects) and conceptual understanding (proxemics theory, social navigation principles).
    \item \textbf{Laboratory Task Performance} -- Structured tasks requiring students to 
    achieve specific navigation objectives by configuring and tuning social navigation 
    algorithms. Task performance is measured through objective metrics (success rate, efficiency) 
    and process measures (approach, testing strategy).
    \item \textbf{Design Challenge Evaluation} -- Open-ended design challenges requiring 
    students to develop novel solutions to specific social navigation problems. Evaluation 
    criteria include technical correctness, creativity, effectiveness, and justification quality.
    \item \textbf{Reflective Analysis} -- Guided reflection assignments where students 
    analyze the behavior of their navigation solutions and connect observations to 
    theoretical principles. These reflections are assessed for depth of analysis, 
    connection to theory, and evidence-based reasoning.
\end{itemize}
Educational assessments were conducted with undergraduate robotics students (n=24) and 
graduate-level human-robot interaction students (n=17) using the simulation environment 
under controlled classroom conditions.


%----------------------------------------------------------------------------------------
%	SECTION 3
%----------------------------------------------------------------------------------------


\section{Evaluation Criteria}
\label{sec:evaluation_criteria}
We established comprehensive evaluation criteria that address both technical performance 
and social dimensions of navigation, as well as educational effectiveness.

%-----------------------------------
%	SUBSECTION 1
%-----------------------------------

\subsection{Technical Performance Criteria}
\label{subsec:technical_performance}
Technical performance was evaluated against the following criteria:
\begin{itemize}
    \item \textbf{Navigation Success Rate} -- The percentage of trials where the robot 
    successfully reaches its goal position without collisions or deadlocks. This primary
     measure of navigation competence is calculated across scenario categories with 
     increasing difficulty thresholds for more complex environments.
    \item \textbf{Path Efficiency} -- Measured as the ratio between actual path length 
    and optimal path length (with no humans present). Efficiency thresholds were 
    established for different environmental conditions, with the understanding that 
    social navigation necessarily involves detours from the geometrically optimal path.
    \item \textbf{Computational Efficiency} -- Planning cycle times must remain within 
    real-time constraints (planning cycle < 100ms) to ensure responsive navigation. We 
    also evaluated planning stability through analysis of planning cycle time variance.
    \item \textbf{Robustness to Uncertainty} -- Performance degradation under sensor 
    noise, human behavior unpredictability, and environmental variability. We systematically 
    increased these uncertainty factors and measured the impact on navigation success 
    and efficiency.
\end{itemize}
These criteria establish minimum thresholds for technical acceptability while recognizing 
the inherent trade-offs involved in social navigation. Rather than optimizing for a single 
technical metric, we seek balanced performance across the technical criteria while 
maintaining social compliance.


%-----------------------------------
%	SUBSECTION 2
%-----------------------------------

\subsection{Social Compliance Criteria}
\label{subsec:social_compliance}
Social compliance was evaluated against both general social navigation principles and 
scenario-specific social conventions:
\begin{itemize}
    \item \textbf{Proxemic Compliance} -- Adherence to established proxemic zones around 
    humans, measured through zone penetration frequency, duration, and depth. We established 
    graduated compliance thresholds based on environmental constraints, with more stringent 
    requirements in unconstrained spaces.
    \item \textbf{Behavioral Legibility} -- The predictability and understandability of robot 
    motion from a human perspective. This is assessed through trajectory analysis 
    (acceleration patterns, path consistency) and expert evaluation of motion naturalness.
    \item \textbf{Minimization of Human Disturbance} -- The degree to which robot navigation 
    avoids disrupting human trajectories or causing humans to change their preferred paths. 
    This is measured through analysis of human trajectory deviations in the presence vs. 
    absence of the robot.
    \item \textbf{Adherence to Social Conventions} -- Compliance with contextual social 
    norms including:
    \begin{itemize}
        \item Corridor passing: Consistent side selection, appropriate speed modulation
        \item Corner navigation: Cautious approach, clear intent signaling
        \item Group navigation: Treating coherent groups as units, avoiding group separation
    \end{itemize}
\end{itemize}
We recognize that social compliance criteria must be context-sensitive, with different 
expectations in different environmental and cultural contexts. Our evaluation framework 
accommodates these contextual factors through parameterized thresholds that can be adjusted 
to reflect specific social contexts.

%-----------------------------------
%	SUBSECTION 3
%-----------------------------------

\subsection{Educational Effectiveness Criteria}
\label{subsec:educational_effectiveness}
Educational effectiveness was evaluated against learning objectives at multiple levels:
\begin{itemize}
    \item \textbf{Knowledge Acquisition} -- Students should demonstrate understanding of:
    \begin{itemize}
        \item Proxemic theory and its application to robotics
        \item Social cost function formulations and effects
        \item Parameter-behavior relationships in social navigation
        \item Evaluation methods for social navigation performance
    \end{itemize}
    \item \textbf{Skill Development} -- Students should develop abilities to:
    \begin{itemize}
        \item Configure and tune social navigation algorithms for specific contexts
        \item Diagnose and resolve navigation issues through systematic parameter adjustment
        \item Implement and modify social cost functions
        \item Design effective experiments to evaluate navigation performance
    \end{itemize}

    \item \textbf{Conceptual Understanding} -- Students should form mental models that:
    \begin{itemize}
        \item Connect algorithm behavior to human social expectations
        \item Recognize the inherent trade-offs in social navigation objectives
        \item Anticipate how environmental factors influence appropriate social behavior
        \item Apply social navigation principles across different robotic contexts
    \end{itemize}
\end{itemize}
These educational criteria were assessed through a combination of direct measures (knowledge 
tests, performance tasks) and indirect measures (reflective assignments, experimental design 
quality).


%----------------------------------------------------------------------------------------
%	SECTION 3
%----------------------------------------------------------------------------------------

\section{Analysis Methods}
\label{sec:analysis_methods}
Our analysis approach combines statistical methods, trajectory analysis techniques, and 
qualitative interpretation to derive meaningful insights from the experimental data.

%-----------------------------------
%	SUBSECTION 1
%-----------------------------------
\subsection{Statistical Analysis Framework}
\label{subsec:statistical_analysis}
We employed the following statistical methods to analyze experimental results:
\begin{itemize}
    \item \textbf{Hypothesis Testing} -- We formulated specific hypotheses about algorithm 
    performance and social compliance, testing these hypotheses using appropriate statistical 
    tests (t-tests for pairwise comparisons, ANOVA for multi-factor analysis). Significance 
    threshold was set at p < 0.05 with Bonferroni correction for multiple comparisons.
    \item \textbf{Parameter Sensitivity Analysis} -- To understand the relationship between 
    algorithm parameters and navigation outcomes, we conducted sensitivity analysis using 
    response surface methodology. This approach maps the parameter space to outcome metrics, 
    identifying critical parameters and optimal operating regions.
    \item \textbf{Multi-objective Performance Analysis} -- Given the inherent trade-offs 
    between efficiency, safety, and social compliance, we employed multi-objective analysis 
    techniques including Pareto frontier identification to characterize the performance 
    envelope of different navigation approaches.
    \item \textbf{Learning Analytics} -- For educational assessment, we applied learning 
    analytics techniques including knowledge gain analysis (normalized pre/post differences), 
    skill development trajectories, and correlation analysis between different assessment 
    dimensions.
\end{itemize}
Statistical analysis was performed using R (v4.2.0) with specialized packages for 
robotics data analysis and visualization (TrajR, HRIstats).

%-----------------------------------
%	SUBSECTION 2
%-----------------------------------

\subsection{Trajectory Analysis}
\label{subsec:trajectory_analysis}
We developed specialized trajectory analysis methods to extract meaningful social navigation 
metrics:
\begin{itemize}
    \item \textbf{Social Interaction Identification} -- Automated detection of interaction 
    episodes within trajectory data, identifying when the robot and humans are engaged in 
    navigation coordination. This detection uses proximity thresholds, relative velocity, 
    and heading alignment to segment continuous trajectories into discrete interaction events.
    \item \textbf{Navigation Strategy Classification} -- Machine learning based classification 
    of robot navigation strategies (e.g., proactive avoidance, reactive adjustment, human-following) 
    from trajectory features. This classification was trained on expert-labeled trajectory segments 
    and achieves 87\% agreement with human coders.
    \item \textbf{Quality of Interaction Metrics} -- Derived measures that combine multiple trajectory 
    features to assess interaction quality, including fluency (smoothness of interleaved trajectories), 
    coordination (mutual adaptation patterns), and minimal jerk criteria.
\end{itemize}
Trajectory analysis was implemented through custom Python libraries built on NumPy, SciPy, and 
specialized robotics analysis tools, integrated with ROS through dedicated analysis nodes.

%-----------------------------------
%	SUBSECTION 3
%-----------------------------------

\subsection{Qualitative Analysis}
\label{subsec:qualitative_analysis}
For qualitative data from expert evaluations and student reflections, we implemented a 
structured analysis approach:
\begin{itemize}
    \item \textbf{Thematic Analysis} -- Identification of recurring themes in qualitative data 
    using an initial codebook developed from theoretical principles, expanded through inductive 
    coding. Code application was performed by multiple coders with regular inter-coder 
    reliability assessment.
    \item \textbf{Critical Instance Analysis} -- Detailed examination of particularly successful 
    or problematic navigation episodes, combining trajectory data, algorithm state information, 
    and expert assessment to understand the factors contributing to the outcome.
    \item \textbf{Comparative Case Analysis} -- Structured comparison of navigation behavior 
    across different algorithms, environmental conditions, and human behavior models to identify 
    patterns and principles that generalize across contexts.
\end{itemize}
Qualitative analysis was conducted using MAXQDA software with a team-based coding 
approach that combines robotics expertise and human-robot interaction knowledge.

%-----------------------------------
%	SUBSECTION 4
%-----------------------------------

\section{Methodological Limitations}
\label{sec:limitations}
We acknowledge several limitations in our methodology that must be considered 
when interpreting results:
\begin{itemize}
    \item \textbf{Simulation Fidelity Limitations} -- While Arena-Rosnav provides 
    high-fidelity simulation, it cannot perfectly reproduce the complexity of real-world 
    human behavior, particularly the subtle social signals and cultural variations present 
    in human navigation. Our results should be interpreted with this limitation in mind.
    \item \textbf{Human Model Simplifications} -- The human agent models, though based 
    on empirical human movement data, necessarily simplify human decision-making and 
    responsiveness. In particular, the models may not fully capture the adaptive and 
    strategic aspects of human navigation behavior in response to robots.
    \item \textbf{Scenario Coverage} -- While we have designed scenarios to represent 
    common navigation contexts, they cannot exhaustively cover the full range of social 
    navigation challenges. Results may not generalize to substantially different 
    environments or interaction contexts.
    \item \textbf{Educational Population Limitations} -- The educational assessment was 
    conducted with computer science and robotics students who may have different prior 
    knowledge and learning approaches than other potential user groups for social 
    navigation education.
\end{itemize}
We have attempted to mitigate these limitations through careful experimental design, 
validation of simulation components against real-world data where possible, and 
appropriate qualification of conclusions based on methodological constraints.


%----------------------------------------------------------------------------------------
%	SECTION 3
%----------------------------------------------------------------------------------------


\section{Summary}
\label{sec:methodology_summary}
This chapter has outlined a comprehensive methodological framework for evaluating 
social navigation algorithms both technically and socially while assessing their 
educational value. Our approach extends the capabilities of the Arena-Rosnav 
simulation environment with specialized data collection, analysis, and evaluation 
methods tailored to social navigation research.

The experimental design provides systematic coverage of navigation scenarios with 
controlled progression of complexity, while the data collection system captures 
both quantitative metrics and qualitative aspects of navigation performance. Our 
evaluation criteria balance technical performance with social compliance 
considerations, recognizing that effective social navigation requires optimization 
across multiple, sometimes competing objectives.

The analysis methods combine statistical rigor, specialized trajectory analysis, 
and structured qualitative approaches to derive meaningful insights from complex 
navigation data. While acknowledging methodological limitations, this approach 
provides a solid foundation for the experimental results presented in subsequent 
chapters.

This methodology supports both the scientific objectives of advancing social navigation 
algorithms and the educational goals of developing effective teaching approaches for 
social navigation concepts. By integrating these dual purposes into a unified 
methodological framework, we contribute not only to the development of better 
navigation algorithms but also to the training of future roboticists who will 
implement these algorithms in real-world systems.