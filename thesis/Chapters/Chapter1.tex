% Chapter Template

\chapter{Introduction} % Main chapter title

\label{Chapter1} % Change X to a consecutive number; for referencing this chapter elsewhere, use \ref{ChapterX}

%----------------------------------------------------------------------------------------
%	SECTION 1
%----------------------------------------------------------------------------------------

\section{Background and Motivation}
% Robot navigation in human environments requires more than obstacle avoidance;
% it demands social awareness and adherence to human proxemic norms. While 
% traditional navigation focuses on collision avoidance and path optimization, 
% social navigation must account for human comfort, cultural norms, and implicit 
% social rules. Socially-aware robot interaction is a very complex topic and 
% interdisciplinary by design. It covers robot navigation tasks, social and 
% cultural rules as well as Human Behaviour Analysis \cite{MOLLERsurvey}. 
% The development of such mobile social robots poses two main challenges: first, 
% real experimentation with people is costly and can be dangerous mainly during
% the initial stages of development, second, difficult to perform and to replicate
% except for very limited, controlled scenarios. Recent years have seen strides in 
% social robotics \cite{arenarosnav}, \cite{hunavsimros2human}, \cite{habitat} 
% with numerous research works proposing platforms and approaches for social 
% navigation and benchmarking. 

As robots increasingly operate in human-populated environments such as hospitals, 
shopping malls, and homes, traditional navigation approaches focused solely on 
collision avoidance and path optimization have proven insufficient. 
Socially-aware navigation represents a critical advancement that extends 
beyond obstacle avoidance to incorporate human comfort, cultural norms, 
and implicit social conventions. This evolution is essential for robots to gain 
acceptance in shared human spaces.
Social navigation is inherently interdisciplinary, bridging robotics, social 
psychology, cultural anthropology, and human-computer interaction. At its core 
lies the concept of proxemics—the study of human use of space and its cultural 
variations—first introduced by anthropologist Edward T. Hall in 1966. Hall's 
delineation of intimate, personal, social, and public interaction zones provides 
a fundamental framework for robot navigation in human environments, as respecting 
these invisible boundaries is crucial for social acceptance.
The development of socially-aware mobile robots faces several significant challenges:
\begin{itemize}
    \item \textit{Complexity of Social Rules} - Human spatial behavior is governed by implicit, 
    context-dependent rules that vary across cultures and situations. These rules are 
    typically learned intuitively by humans but must be explicitly encoded for robots.
    \item \textit{Experimentation Challenges} - Real-world testing with humans is 
    resource-intensive, potentially risky during early development stages, and 
    difficult to reproduce consistently across different research groups.
    \item \textit{Interdisciplinary Knowledge Requirements} - Developing effective social navigation 
    systems demands expertise across multiple domains, creating both research and 
    educational barriers.
    \item \textit{Technical Implementation Hurdles} - Integrating social awareness into existing 
    navigation frameworks requires sophisticated software architecture and computational 
    efficiency to maintain real-time performance.
\end{itemize}

Recent advances in social robotics have produced numerous platforms and approaches for 
addressing these challenges (\cite{arenarosnav}, \cite{hunavsimros2human}, \cite{habitat}). 
However, these innovations often remain siloed within research laboratories, with limited 
accessibility for educational purposes. This accessibility gap represents a significant 
obstacle for preparing the next generation of roboticists to develop socially-aware systems.
%----------------------------------------------------------------------------------------
%	SECTION 2
%----------------------------------------------------------------------------------------

\section{Problem Statement}
Often times, navigation systems treat humans as mere dynamic obstacles without 
considering social contexts, leading to behaviors that may be technically efficient 
but socially inappropriate. This research addresses the gap between technically 
sound and socially acceptable robot navigation in human-shared spaces. 

Despite significant research advances in social navigation, three critical problems persist:
\begin{enumerate}
    \item \textit{Technical-Social Disconnect} - Most deployed robot navigation systems continue to treat 
    humans as mere dynamic obstacles, disregarding social contexts and norms. This approach 
    leads to behaviors that may be computationally efficient but socially inappropriate or 
    discomforting to humans sharing the space.
    \item \textit{Educational Access Barriers} - Existing social navigation implementations typically demand 
    extensive technical expertise and computational resources, making them inaccessible for 
    educational purposes. Students face significant hurdles in learning about and 
    experimenting with social navigation concepts due to complex installation requirements, 
    dependencies, and hardware constraints.
    \item \textit{Lack of Standardized Learning Tools} - While several research platforms exist, there is a 
    notable absence of standardized educational tools that demonstrate social navigation 
    concepts in a structured, pedagogically sound manner. This gap impedes effective 
    teaching and learning of this increasingly important aspect of robotics.
\end{enumerate}

This thesis addresses these interconnected problems by developing an accessible, education-focused platform that bridges the gap between advanced social navigation research and practical robotics education.
%----------------------------------------------------------------------------------------
%	SECTION 3
%----------------------------------------------------------------------------------------

\section{Research objectives}
% As part of the Bachelor thesis, the system will be developed with educational use in mind.
% The goal is to create a testbench for social robot navigation that allows to be employed 
% both reserach and teaching purposes: a virtual laboratory experiment that a) demonstrate 
% a variety of off-the-shelf social navigation algorithms, b) show the effect of user-defined
% custom cost functions and/or evaluation metrics used in navigation algorithms, c) modify 
% human behaviour simulation, and (optionally / ideally) d) run custom social navigation 
% algorithms. The following subsections define the tasks in detail.

This bachelor thesis aims to develop an educational system for teaching and experimenting 
with social robot navigation concepts. The primary goal is to create a virtual laboratory 
environment that lowers technical barriers and provides structured learning experiences. 
Specific objectives include:

\begin{enumerate}
    \item \textbf{Create an accessible virtual testbench} that demonstrates various social navigation concepts 
    without requiring extensive technical setup or specialized hardware. This environment will:
    \begin{itemize}
        \item Integrate existing open-source social navigation implementations
        \item Provide a virtualized environment that runs efficiently on standard student hardware
        \item Offer a unified interface for interaction with different navigation approaches
    \end{itemize}
    
    \item \textbf{Develop structured educational experiments} that:
    \begin{itemize}
        \item Demonstrate a variety of off-the-shelf social navigation algorithms
        \item Illustrate the effects of different proxemic models and parameter configurations
        \item Allow comparison between socially-aware and traditional navigation approaches
        \item Showcase the impact of environmental context on navigation behavior
    \end{itemize}
    
    \item \textbf{Enable hands-on experimentation} through:
    \begin{itemize}
        \item Real-time costmap parameter modification
        \item Customizable evaluation metrics for navigation performance
        \item Visualization tools for understanding algorithm decision-making
    \end{itemize}
    
    \item \textbf{Create comprehensive educational materials} including:
    \begin{itemize}
        \item Laboratory exercises with clear learning objectives
        \item Documentation explaining theoretical concepts and their implementation
        \item Guided exploration activities with progressive complexity
    \end{itemize}
\end{enumerate}


The system is designed primarily for undergraduate and graduate robotics courses, 
enabling students to develop an intuitive understanding of social navigation 
principles through direct experimentation before potentially developing their 
own implementations.
%----------------------------------------------------------------------------------------
%	SECTION 4
%----------------------------------------------------------------------------------------

\section{Scope and Limitations}

This thesis focuses on creating an educational platform for social navigation rather than developing novel navigation algorithms or proxemic models. The scope encompasses:

\subsection*{In Scope:}
\begin{itemize}
    \item Integration of existing open-source social navigation implementations
    \item Development of a virtualized environment for accessible deployment
    \item Creation of standardized test scenarios for comparative evaluation
    \item Design of structured educational experiments and supporting materials
    \item Implementation of visualization tools for algorithm behavior and decision processes
    \item Extension of existing platforms with missing components required for educational purposes
\end{itemize}

\subsection*{Out of Scope:}
\begin{itemize}
    \item Development of fundamentally new social navigation algorithms
    \item Large-scale human studies to validate navigation approaches
    \item Physical robot implementation and testing
    \item Cross-platform compatibility beyond the specified virtualization approach
    \item Comprehensive cultural adaptation of proxemic models
\end{itemize}

\subsection*{Limitations:}
\begin{itemize}
    \item The system will prioritize educational clarity over computational performance
    \item Simulated human behaviors will represent simplified models of actual human movement patterns
    \item The virtualized environment introduces some performance overhead compared to native installation
    \item The platform targets educational use cases rather than deployment-ready implementations
\end{itemize}

These scope boundaries ensure the project remains achievable within the constraints of a 
bachelor thesis while still delivering significant educational value through an accessible 
platform for teaching social navigation concepts.