% Chapter Template

\chapter{System Design and Implemenation} % Main chapter title
This chapter presents a comprehensive overview of the system design and implementation details of 
our social navigation educational platform. We begin with a thorough examination of the simulation 
environment, including the rationale behind our platform selection and the architectural design 
choices. We then detail the specific customizations made to enhance the educational value of the 
system. Finally, we describe the implementation of various social navigation scenarios, focusing 
on their design, challenges, and parameterization options.

\label{Chapter3} % Change X to a consecutive number; for referencing this chapter elsewhere, use \ref{ChapterX}

%----------------------------------------------------------------------------------------
%	SECTION 1
%----------------------------------------------------------------------------------------

\section{Overview of the Simulation Environment}

The simulation environment forms the foundation of our educational platform for social robot navigation. 
Its design required careful consideration of educational objectives, technical capabilities, and 
accessibility for students with varying levels of robotics experience. After evaluating several 
options, we selected Arena-Rosnav as our primary simulation platform.

%-----------------------------------
%	SUBSECTION 1
%-----------------------------------
\subsection{Platform Selection and Justification}
\label{sec:platform_selection_and_justification}

The selection of Arena-Rosnav as our simulation environment was based on a systematic evaluation of 
requirements for both research-grade social navigation experiments and effective educational tools. 
We identified several critical requirements:

\begin{itemize}
    \item \textbf{ROS Integration}: The platform needed to offer seamless integration with the Robot 
    Operating System (ROS) ecosystem, which remains the predominant middleware in robotics research 
    and education.
    \item \textbf{Social Navigation Support}: The environment needed native support for human agents 
    and the ability to model social interactions.

    \item \textbf{Scalability}: The platform should support scenarios with varying complexity, from 
    simple corridor interactions to complex multi-agent environments.

    \item \textbf{Reproducibility}: The environment needed to provide deterministic simulation capabilities 
    for scientific reproducibility and consistent educational experiences.

    \item \textbf{Extensibility}: The platform should allow for straightforward customization and extension 
    to incorporate new navigation algorithms and social behavior models.

    \item \textbf{Accessibility}: The simulation environment needed to be approachable for undergraduate 
    students while still offering the depth required for graduate-level research.
\end{itemize}

Arena-Rosnav \cite{arena30} satisfies these requirements with its comprehensive feature set. Specifically, 
Arena-Rosnav provides:

\begin{enumerate}
\item \textbf{Modular Architecture}: Arena-Rosnav offers a highly modular architecture that separates robot 
models, navigation algorithms, and environmental factors. This modularity allows students to focus on specific 
components without needing to understand the entire system at once.
\item \textbf{Multiple Robot Support}: The platform supports various robot models (including TurtleBot3, 
Jackal, and custom platforms), allowing for experimentation with different kinematic constraints and 
sensor configurations without changing the underlying navigation code.

\item \textbf{Advanced Human Simulation}: Arena-Rosnav implements the Social Force Model (SFM) 
and ORCA (Optimal Reciprocal Collision Avoidance) for human agent movement, creating realistic crowd behaviors. 
This allows for more authentic testing of social navigation algorithms compared to environments with simplistic 
human models.

\item \textbf{Scenario Generation}: The platform includes tools for procedural generation of environments 
and scenarios, which facilitates systematic testing across a broad spectrum of conditions. This feature 
is particularly valuable for our educational objectives, as it allows students to test navigation algorithms 
against consistent challenges.

\item \textbf{Benchmarking Tools}: Arena-Rosnav provides built-in performance metrics and visualization 
tools that align well with our requirement for educational clarity and research rigor.

\item \textbf{Active Community}: The platform has an active development community, ensuring longevity 
and continuous improvement of the codebase.

\item \textbf{Open-Source Accessibility}: As an open-source project, Arena-Rosnav allows for complete 
transparency and modification, which is essential for both research innovation and educational customization.
\end{enumerate}
Other platforms considered included HuNavSim and Habitat-Sim. While these alternatives offered some advantages, 
Arena-Rosnav provided the most comprehensive 
feature set for our specific needs, particularly in human-robot interaction scenarios and educational 
flexibility. Gazebo standalone environments lacked standardized human models and would have required significant 
development to reach feature parity with Arena-Rosnav. Stage simulator, while lightweight, lacks the physical 
fidelity needed for realistic social navigation research. Custom environments would have required prohibitive 
development time without offering significant advantages over the already mature Arena-Rosnav platform.

Furthermore, Arena-Rosnav's recent updates have enhanced its crowd simulation capabilities, with implementations 
of the latest pedestrian modeling approaches and improved computational efficiency that makes it suitable 
for running on standard laboratory computers available to students.


%-----------------------------------
%	SUBSECTION 2
%-----------------------------------

\subsection{System Architecture}
The architecture of our simulation environment builds upon the core Arena-Rosnav structure while integrating 
additional components specific to social navigation education and research. The system architecture, 
illustrated in Figure \ref{fig:system_architecture}, consists of five primary layers:

\begin{figure}[h]
    \centering
    % Placeholder for system architecture diagram
    \caption{System architecture of the social navigation simulation environment}
    \label{fig:system_architecture}
\end{figure}

\subsubsection{Core Simulation Layer}
At the foundation of our architecture is the core simulation layer, which provides the physical 
simulation environment. This layer:
\begin{itemize}
    \item Utilizes Gazebo as the underlying physics engine, providing realistic simulation of 
    robot dynamics, sensor interactions, and environmental physics.
    \item Implements a temporal discretization model that balances simulation fidelity with 
    computational efficiency, using a variable time-step approach that allocates more 
    computational resources to complex interaction scenarios.
    \item Provides environmental rendering through both Gazebo's built-in visualization 
    and custom RViz configurations optimized for observing social navigation behaviors.
    \item Manages simulation state, including robot poses, human positions and velocities, 
    and environmental object states, all synchronized through ROS topic communication.
\end{itemize}

\subsubsection{Human Simulation Layer}
The human simulation layer is responsible for generating realistic pedestrian behaviors:
\begin{itemize}
    \item Implements multiple pedestrian models including SFM (Social Force Model), which simulates humans 
    as particles influenced by social and physical forces, and ORCA (Optimal Reciprocal Collision Avoidance), 
    which provides more computationally efficient local navigation for large crowds.
    \item Models human intentions through goal-directed behavior with configurable parameters 
    for walking speed, personal space preferences, and decision-making characteristics.
    \item Includes crowd formation models that can simulate common social groupings (pairs, families, 
    tour groups) with appropriate inter-personal spacing and coordination behaviors.
    \item Provides a human behavior API that allows for scripted scenarios where human agents follow 
    predetermined paths or respond to environmental triggers, essential for reproducible educational 
    experiments.
\end{itemize}

\subsubsection{Robot Navigation Layer}
The robot navigation layer encapsulates the navigation stack components:
\begin{itemize}
    \item Integrates the standard ROS navigation stack with customizable costmap layers, 
    including traditional layers (static, obstacle, inflation) and social navigation 
    layers (proxemics, passing).
    \item Implements multiple local planners, including the default Dynamic Window 
    Approach (DWA) planner, Timed Elastic Band (TEB) planner, and custom social 
    planners developed for this research.
    \item Provides standardized interfaces for global planners, including A*, 
    Dijkstra, and RRT variants, with hooks for social cost integration at the 
    global planning level.
    \item Exposes navigation parameters through both ROS parameter server and structured 
    configuration files, facilitating systematic experimentation by students.
\end{itemize}

\subsubsection{Social Interaction Layer}
The social interaction layer, which is our primary contribution to the Arena-Rosnav 
architecture, implements social navigation concepts:
\begin{itemize}
    \item Provides the proxemics costmap layer that dynamically generates cost fields 
    around humans based on proxemic theories of personal, social, and public space.
    \item Implements the passing layer that models directional preferences when 
    navigating around humans, incorporating cultural and contextual variables.
    \item Includes visualization components that render proxemic zones, anticipated 
    human trajectories, and robot planning decisions in human-interpretable formats.
    \item Offers a social metrics collector that measures and records quantities such 
    as minimum separation distance, personal space intrusions, and trajectory 
    smoothness during navigation.
\end{itemize}

\subsubsection{Educational Interface Layer}
The educational interface layer makes the system accessible to students:
\begin{itemize}
    \item Provides simplified launch files with graduated complexity, allowing students 
    to start with basic scenarios and progressively engage with more complex system 
    components.
    \item Implements parameter templates for different experimental configurations, 
    reducing the learning curve for system customization.
    \item Includes data collection scripts that automatically store and process 
    experimental results in formats suitable for analysis and report generation.
    \item Offers debug visualization through custom RViz configurations that highlight 
    relevant aspects of social navigation for educational understanding.
\end{itemize}
Data flow through this architecture follows a consistent pattern. Environmental perception 
(simulated sensor data) flows from the core simulation layer to the robot navigation layer. 
The human simulation layer provides human position and velocity data to both the core 
simulation layer and the social interaction layer. The social interaction layer processes 
human data to generate social costs, which are then integrated into the navigation layer's 
planning processes. The educational interface layer primarily consumes data from other 
layers for visualization and analysis while providing configuration inputs that modify 
layer behaviors.

This multi-layered architecture provides both the technical depth required for 
research-grade experimentation and the conceptual clarity needed for educational 
applications. The clear separation of concerns between layers allows students to focus 
on specific aspects of social navigation without being overwhelmed by system complexity.


%----------------------------------------------------------------------------------------
%	SECTION 2
%----------------------------------------------------------------------------------------

\section{Customization for Educational Use}

While Arena-Rosnav provides an excellent foundation for social navigation research, 
significant customization was necessary to optimize it for educational purposes. Our 
customizations focused on reducing complexity barriers, providing educational scaffolding, 
and enhancing the visualization of social navigation concepts.

\subsection{Simplified Configuration System}
We developed a tiered configuration system that allows students to engage with increasing 
levels of complexity:
\begin{itemize}
    \item \textbf{Level 1: Parameter Templates} - Predefined parameter sets that 
    allow students to run complete experiments without parameter tuning, focusing
    instead on observing and analyzing results.
    \item \textbf{Level 2: Guided Configuration} - Template files with clearly documented 
    parameters that students can modify within safe ranges, encouraging experimentation 
    while preventing system failures due to invalid configurations.
    \item \textbf{Level 3: Full Configuration} - Complete access to all system parameters 
    for advanced students and research projects, with extensive documentation of parameter 
    interactions and effects.
\end{itemize}
This tiered approach enables instructors to match system complexity to student preparation 
and learning objectives. For introductory robotics courses, Level 1 configurations allow 
students to focus on conceptual understanding. For advanced courses, Level 3 configurations 
promote deeper technical engagement.

\subsection{Enhanced Visualization Tools}
We developed specialized visualization components that make social navigation concepts 
visually explicit:
\begin{itemize}
    \item \textbf{Proxemic Visualization} - Custom RViz plugins that render personal, social, 
    and public spaces around simulated humans as color-coded regions with appropriate 
    transparency to maintain visibility of the underlying environment.
    \item \textbf{Planning Visualization} - Visualization tools that show considered and 
    rejected paths, highlighting how social costs influence navigation decisions.
    \item \textbf{Cost Function Visualization} - Heat map displays that show the distribution 
    of navigation costs throughout the environment, making the abstract concept of cost 
    functions tangible for students.
    \item \textbf{Metric Displays} - Real-time displays of key social navigation metrics 
    (minimum separation distance, personal space intrusions, path efficiency) that allow 
    students to immediately observe the effects of parameter changes.
\end{itemize}
These visualization enhancements are crucial for educational effectiveness, as they make abstract 
algorithmic concepts visually concrete. Students can directly observe how proxemic models 
influence robot behavior, building intuition that complements their theoretical understanding.

\subsection{Structured Learning Scenarios}
We developed a progression of learning scenarios that introduce social navigation concepts 
in a structured sequence:
\begin{itemize}
    \item \textbf{Observation Scenarios} - Initial scenarios where students observe 
    pre-configured robots navigating around humans, focusing on developing observational 
    skills before implementation.
    \item \textbf{Parameter Exploration Scenarios} - Scenarios where students modify specific 
    parameters and observe outcomes, building cause-effect understanding of navigation algorithms.
    \item \textbf{Comparative Analysis Scenarios} - Scenarios that run multiple navigation 
    approaches simultaneously for side-by-side comparison, highlighting the advantages and 
    limitations of different approaches.
    \item \textbf{Challenge Scenarios} - Complex environments that test student understanding 
    through navigation problems requiring thoughtful parameter selection and algorithm choice.
\end{itemize}
These structured scenarios provide a pedagogical progression that aligns with educational 
best practices for skills development. Students move from guided observation to independent 
problem-solving, developing both technical skills and conceptual understanding.

\subsection{Documentation and Tutorials}
We developed comprehensive educational materials specifically designed for the learning 
progression:
\begin{itemize}
    \item \textbf{Conceptual Tutorials} - Materials explaining the theoretical foundations of 
    social navigation, proxemics theory, and human-aware planning.
    \item \textbf{Technical Walkthroughs} - Step-by-step guides for system installation, configuration, 
    and operation, accommodating students with varying technical backgrounds.
    \item \textbf{Experimental Guides} - Structured procedures for conducting experiments, including 
    data collection protocols, analysis methods, and interpretation guidelines.
    \item \textbf{Troubleshooting Resources} - Common error documentation and resolution strategies, 
    reducing the frustration factor often associated with complex simulation environments.
\end{itemize}
Unlike general-purpose documentation, these materials are explicitly aligned with educational 
objectives and student developmental progression. They incorporate pedagogical best practices 
such as scaffolded learning, worked examples, and conceptual bridges between theory and 
implementation.
%----------------------------------------------------------------------------------------
%	SECTION 3
%----------------------------------------------------------------------------------------

\section{Implementation of Social Navigation Scenarios}
The effectiveness of our simulation environment depends significantly on the quality of the 
implemented scenarios. We developed specific scenarios that highlight key aspects of social 
navigation while providing controlled experimental conditions for both research and education.
%-----------------------------------
%	SUBSECTION 1
%-----------------------------------

\subsection{Corridor Scenario Implementation}
The corridor scenario represents one of the most common human-robot interaction contexts 
in indoor environments. This scenario tests a robot's ability to navigate in confined spaces 
while respecting human proxemic preferences.

\subsubsection{Physical Environment Design}
The corridor environment is implemented with the following characteristics:
\begin{itemize}
    \item \textbf{Dimensions} - The corridor measures 2.5 meters in width and 15 meters in 
    length, representing a typical institutional hallway dimension that is narrow enough to 
    force interaction decisions but wide enough to allow multiple navigation strategies.
    \item \textbf{Boundary Definition} - Corridor walls are defined as both collision objects 
    in the physics engine and as occupied cells in the static costmap layer, ensuring consistent 
    representation across the simulation stack.
    \item \textbf{Surface Properties} - The corridor floor uses a friction coefficient that 
    accurately models typical indoor flooring, ensuring realistic robot motion dynamics, 
    particularly for differential drive platforms that may experience wheel slip during 
    rapid direction changes.
    \item \textbf{Contextual Elements} - The environment includes typical corridor features 
    such as doorways, wall fixtures, and occasional side tables that add environmental complexity 
    without fundamentally changing the navigation challenge.
\end{itemize}
The physical dimensions were carefully selected based on architectural standards and 
empirical studies of human comfort in corridor passing interactions. The width of 2.5 meters 
creates what proxemics researchers call a "decision zone" where robots must make explicit 
passing choices rather than simply maintaining maximum separation.

\subsubsection{Human Agent Configuration}
The human agents in the corridor scenario are configured to create realistic navigation 
challenges:
\begin{itemize}
    \item \textbf{Movement Patterns} - Humans follow bidirectional paths along the corridor with 
    randomized entry timing to create varied encounter scenarios (oncoming, overtaking, group passing).
    \item \textbf{Walking Characteristics} - Human walking speeds are drawn from a normal 
    distribution with mean 1.4 m/s and standard deviation 0.2 m/s, based on empirical 
    human walking speed studies.
    \item \textbf{Group Formations} - The scenario includes both individual pedestrians and 
    small groups (pairs and triads) that maintain social formation during navigation, 
    presenting more complex spatial challenges.
    \item \textbf{Attention Models} - Humans are configured with variable attention states, 
    including attentive (aware of and responsive to the robot) and inattentive 
    (minimally responsive), reflecting the range of human behaviors in real environments.
\end{itemize}
These configurations create a rich variety of interaction scenarios while maintaining enough 
consistency for experimental repeatability. The human models were calibrated against empirical 
observations of pedestrian behavior in corridor environments to ensure behavioral realism.

\subsubsection{Scenario Variants}
We implemented multiple variants of the corridor scenario to isolate specific 
navigation challenges:
\begin{itemize}
    \item \textbf{Basic Passing}     - A single oncoming human in an otherwise empty corridor, 
    testing fundamental passing behavior.
    \item \textbf{Multiple Passing} - Multiple humans moving in both directions, requiring 
    the robot to handle sequential interactions.
    \item \textbf{Group Navigation} - Scenarios featuring human groups that occupy more 
    horizontal space, testing the robot's ability to navigate around collective social entities.
    \item \textbf{Crowded Corridor} - High-density human traffic that approaches the 
    corridor's capacity limit, testing navigation in socially dense environments.
\end{itemize}
These variants allow for systematic investigation of specific social navigation challenges
while maintaining the basic corridor context. They progress from simple to complex 
interactions, supporting both structured education and research depth.
%-----------------------------------
%	SUBSECTION 2
%-----------------------------------

\subsection{Corner Scenario Implementation}
The corner scenario tests a robot's ability to navigate blind corners safely and socially, 
a significant challenge in many indoor environments. This scenario is particularly valuable 
for evaluating how robots handle limited visibility combined with potential human encounters.

\subsubsection{Physical Environment Design}
The corner environment has the following characteristics:
\begin{itemize}
    \item \textbf{Geometry} - A 90-degree corner with 2.5-meter wide corridors on each 
    side, creating a navigation challenge where the robot cannot see around the corner 
    until committed to the turn.
    \item \textbf{Visibility Constraints} - The corner is designed with solid walls that 
    block both visual and laser sensor data, preventing the robot from detecting humans 
    on the other side of the corner until physically approaching the intersection.
    \item \textbf{Approach Zones} - The corridors extend 8 meters from the corner in 
    each direction, providing sufficient distance for the robot to adjust its approach 
    velocity based on corner-handling strategies.
    \item \textbf{Environmental Markers} - Subtle environmental cues (floor coloration 
    changes, wall textures) are included 2 meters before the corner, allowing for the 
    development and testing of anticipatory algorithms that recognize corner contexts.
\end{itemize}
The corner geometry was specifically designed to create the partial observability problem 
that makes corner navigation socially challenging. The 90-degree angle represents the most 
common architectural corner configuration while creating a genuine blind spot that cannot 
be fully resolved with current sensor technologies.

\subsubsection{Human Agent Configuration}
Human agents in the corner scenario are configured to create challenging interaction 
situations:
\begin{itemize}
    \item \textbf{Trajectory Patterns} - Humans approach the corner from multiple 
    directions with paths that create potential collision trajectories if the corner 
    is navigated without social awareness.
    \item \textbf{Variable Speeds} - Human walking speeds are more variable in this 
    scenario (0.8 m/s to 1.8 m/s) to test robot adaptation to different human 
    approach velocities.
    \item \textbf{Corner Behavior Models} - Human agents implement realistic corner 
    navigation behaviors, including trajectory adjustment, speed modulation when 
    approaching blind corners, and recovery behaviors when near-collisions occur.
    \item \textbf{Attention Distribution} - The scenario includes a higher proportion 
    of distracted humans (e.g., looking at mobile devices) who may not actively avoid 
    the robot, creating a more realistic and challenging navigation environment.
\end{itemize}
These human configurations create the challenging corner interactions observed in 
real environments where humans must negotiate shared space with limited advance 
information. The behavior models were validated against observational studies of 
human corner navigation in public buildings.

\subsubsection{Scenario Variants}
We implemented several corner scenario variants to isolate specific navigation challenges:
\begin{itemize}
    \item \textbf{Basic Corner Encounter} - Single humans approaching from around 
    the corner, testing fundamental corner navigation safety.
    \item \textbf{Concurrent Approaches} - Multiple humans approaching the corner 
    simultaneously from different directions, requiring more complex negotiation.
    \item \textbf{Stationary Human} - A stationary human positioned just around the 
    corner, testing the robot's ability to handle sudden static obstacles in socially 
    appropriate ways.
    \item \textbf{Dynamic Speed Adjustment} - Humans who change speed as they approach 
    the corner, testing the robot's adaptation to changing human intentions.
\end{itemize}
These variants systematically explore the corner navigation challenge space, allowing 
for controlled experiments that isolate specific aspects of the problem. The progression 
from basic to complex scenarios supports both educational sequencing and systematic research.

%-----------------------------------
%	SUBSECTION 3
%-----------------------------------

\subsection{Parameterization and Modifiability}

A key requirement for both educational use and research flexibility is comprehensive 
parameterization of the simulation environment. We implemented a multi-level parameter 
system that balances flexibility with usability.

\subsubsection{Physical Environment Parameters}
The physical environments are parameterized through:
\begin{itemize}
    \item \textbf{Dimensional Parameters} - Corridor width, length, corner angles, 
    and approach distances can be modified through configuration files, allowing for 
    systematic exploration of spatial constraints.
    \item \textbf{Object Placement} - Environmental objects (side tables, plants, chairs) 
    can be procedurally placed with controllable density and distribution parameters, 
    enabling systematic variation of environmental complexity.
    \item \textbf{Sensor Noise Models} - Realistic sensor noise can be applied with 
    configurable intensity, allowing students to explore robustness to perceptual 
    uncertainty.
    \item \textbf{Surface Properties} - Floor friction, material reflectivity, and 
    other physical properties can be adjusted to test navigation robustness across 
    different environmental conditions.
\end{itemize}
These parameters allow for systematic manipulation of the physical challenges facing 
the navigation system. The parameterization uses a hierarchical YAML structure that 
organizes related parameters into logical groups, enhancing usability.

\subsubsection{Human Behavior Parameters}
Human agent behavior is highly parameterized:
\begin{itemize}
    \item \textbf{Motion Parameters} - Walking speed distributions, acceleration 
    profiles, and directional variance can be adjusted to model different human 
    movement patterns.
    \item \textbf{Social Parameters} - Personal space preferences, group cohesion 
    forces, and collision avoidance aggressiveness can be modified to represent 
    different cultural and individual behavioral norms.
    \item \textbf{Attentional Parameters} - Human attention to the robot, 
    responsiveness to potential collisions, and path predictability can be 
    adjusted to test robot adaptation to different human awareness levels.
    \item \textbf{Trajectory Generation} - Goal positions, waypoint density, 
    and path constraints can be modified to create different human movement 
    patterns through the environment.
\end{itemize}
These parameters allow for modeling a wide range of human behaviors, from highly 
predictable and robot-aware to unpredictable and inattentive. The parametric models 
are based on established pedestrian simulation research, ensuring behavioral realism.

\subsubsection{Navigation Algorithm Parameters}
The robot navigation stack is comprehensively parameterized:
\begin{itemize}
    \item \textbf{Costmap Parameters} - Each costmap layer (static, obstacle, 
    inflation, proxemics, passing) has adjustable parameters controlling its 
    influence on navigation decisions. Key parameters include:
    \begin{itemize}
        \item Static layer: update frequency, obstacle inflation
        \item Obstacle layer: obstacle detection thresholds, clearing thresholds
        \item Inflation layer: inflation radius, cost scaling factor
        \item Proxemics layer: personal/social/public space radii, cost decay functions
        \item Passing layer: preferred passing side, minimum passing distance
    \end{itemize}
    \item \textbf{Local Planner Parameters} - The DWA and TEB planners 
    expose parameters controlling:
    \begin{itemize}
        \item Path scoring weights (path distance, goal alignment, obstacle clearance)
        \item Kinematic constraints (maximum velocities, accelerations)
        \item Lookahead distance and planning horizon
        \item Social navigation specific weights and constraints
    \end{itemize}
    \item \textbf{Global Planner Parameters} - A*, Dijkstra, and RRT planners expose parameters for:
    \begin{itemize}
        \item Heuristic weighting and behavior
        \item Sampling strategy and density
        \item Path optimization and smoothing
        \item Social cost integration methods
    \end{itemize}
\end{itemize}
These navigation parameters allow for systematic exploration of algorithm behavior 
across a wide range of configurations. The parameters are exposed through both ROS 
parameter server mechanisms and configuration files, supporting both runtime adjustment 
and systematic experimental design.

\subsubsection{Parameter Configuration Interface}
To make this extensive parameterization accessible, we developed a layered 
configuration interface:
\begin{itemize}
    \item \textbf{Scenario Configuration Files} - Top-level files that define complete 
    scenarios, including environmental setup, human behavior patterns, and robot 
    configurations.
    \item \textbf{Component Configuration Files} - Modular files for specific system 
    components (human models, robot navigation, sensors) that can be combined into 
    complete scenarios.
    \item \textbf{Parameter Templates} - Predefined parameter sets representing specific 
    navigation strategies or experimental conditions that can be applied to any scenario.
    \item \textbf{Educational Parameter Guides} - Documentation that explains parameter 
    effects, reasonable ranges, and expected behavioral impacts for student experimentation.
\end{itemize}
This layered approach makes the complex parameter space accessible to users with different 
expertise levels. Students can begin with scenario-level configurations and progressively 
engage with component-level and individual parameters as their understanding develops.

\section{Summary}
This chapter has detailed our system design and implementation choices for creating an 
effective social navigation simulation environment. We selected Arena-Rosnav as our 
foundation due to its comprehensive feature set, particularly its support for realistic 
human simulation and integration with the ROS ecosystem. Our implementation extends this 
platform with specialized components for social navigation research and education.

The system architecture provides clear separation between simulation, human modeling, 
robot navigation, social interaction, and educational interface layers. This modular 
design supports both research flexibility and educational clarity. Our customizations 
for educational use, including simplified configuration systems, enhanced visualizations, 
and structured learning scenarios, make complex social navigation concepts accessible to 
students.

The implemented scenarios—corridor and corner navigation—provide controlled environments 
for investigating key social navigation challenges. These scenarios have been carefully 
designed to balance realism with experimental control, supporting both educational 
objectives and research rigor. The comprehensive parameterization of the environment 
enables systematic exploration of navigation behaviors across a wide range of conditions.

This implementation provides the technical foundation for the student experiments and 
research contributions described in subsequent chapters. By combining research-grade 
simulation capabilities with educational accessibility, our system supports both the 
advancement of social navigation algorithms and the effective teaching of social 
navigation concepts.